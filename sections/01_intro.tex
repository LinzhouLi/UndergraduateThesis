\section{引~言}
\subsection{背景介绍}
\subsubsection{WebGPU}
    如今,浏览器技术日新月异,大量Web APP提供的丰富内容远超“浏览网页”这种简单需求。根据他们的工作负载不同,可以将这些应用程序大致分为控制密集型(如搜索、解析、排序和网络请求)和数据密集型(如图像处理、数值计算、渲染和仿真)两种\cite{K22WebGPU1}。控制密集型应用往往在CPU上运行的更快,而高效地运行数据密集型应用则往往需要大规模并行计算架构处理器——GPU。一直以来,由于软硬件的限制,浏览器环境下并不善于处理大量数据计算。而近两年W3C提出的WebGPU图形API,为Web开发者提供了接近底层的GPU性能与大量现代的图形硬件特性,这使得在浏览器中运行更加富有创造力的应用——复杂流体仿真,成为了可能。
    
    WebGPU的设计目标之一是接替WebGL,这是自浏览器提供GPU支持以来最大规模的图形API架构变动。它给开发者提供了更加自由的硬件资源控制(如支持计算管线,开放通用计算功能), 与更科学的API设计(如指令队列,减少CPU与GPU通信)\cite{K22WebGPU2}。目前,多个3D引擎(Three.js、Babylon.js与Orillusion)都在积极推进对WebGPU的支持。

\subsubsection{实时流体仿真}
    实时流体仿真是计算机图形学领域一个重要的研究方向,涉及模拟流体行为的计算方法和技术。流体在许多领域中起着关键作用,包括游戏开发、动画制作、虚拟现实和工程仿真等。实时流体仿真的目标是以接近实时的速度模拟流体的运动、变形和相互作用,以便在交互式环境中提供逼真的视觉效果和物理反馈。
    
    传统的流体仿真方法通常基于数值求解或网格离散化的技术,如有限元方法或格点流体动力学。然而,这些方法在实时应用中的计算复杂度较高,难以满足实时性的要求。因此,研究者们正在不断探索更高效的实时流体仿真算法。
    
    其中一种被广泛应用的方法是基于粒子的平滑粒子动力学算法(Smoothed Particle Hydrodynamics,SPH)。SPH算法将流体系统离散化为一组粒子,并通过计算粒子之间的相互作用力来模拟流体的行为。SPH算法在处理流体的自由表面、非线性行为和复杂流动时表现出色。然而,由于涉及大量粒子之间的相互作用计算,SPH算法的实时性仍然是一个挑战。
    
    为了克服实时流体仿真的挑战,研究者们开始利用现代计算技术,如图形处理器(GPU)和并行计算,来加速流体仿真算法。而WebGPU正是提供了高性能的并行计算和图形渲染能力。通过结合SPH算法和WebGPU技术,可以实现轻量级的在线复杂流体仿真,使得流体仿真能够在Web浏览器中以接近实时的速度进行。
    
    因此,本毕业设计旨在研究基于WebGPU的轻量级实时流体仿真算法,特别关注SPH算法在Web平台上的加速和优化。通过使用WebGPU的并行计算和高性能图形渲染能力,预计可以改善实时流体仿真的性能和交互性,从而推动在线流体仿真应用的发展。这项研究对于提升实时流体仿真技术的效率和实用性具有重要意义。

\subsection{本文主要工作}
    本文的主要目标是使用WebGPU实现一个轻量级在线3D复杂流体仿真系统,主要工作概况为以下三点:

    \begin{enumerate}
    	\item 设计了一套流体模拟算法。本文使用粒子法的作为流体模拟的主要框架。为了保证不可压缩性,本文使用基于位置的动力学思想迭代求解密度约束,并使用一种隐式边界表示方法——边界体积贴图法实现流体固液耦合。另外,为了进一步提高流体模拟真实感,本文在流体压强求解器之后又加入了三种非压强力的求解模型——表面张力、涡量补偿力与人工粘性力。最后,本文实现了通过空间划分加速的邻域粒子查找算法。
    	\item 设计了一套流体渲染算法。针对粒子法流体仿真的结果,本文提出了一种高效的流体渲染方案。首先将粒子光栅化得到深度图和厚度图,再对深图进行窄域滤波,最后进行屏幕空间渲染。
    	\item 使用WebGPU实现了流体仿真系统。本文将模拟与渲染算法完全并行化,使用WebGPU实现。本仿真系统可以实时运行十万个流体粒子的仿真规模,并完成实时高质量渲染。经过多个仿真场景的实际检验,可以证明仿真系统达到了设计目标。
    \end{enumerate}

\subsection{本文组织结构}
    本文共有八个章节,本文主要介绍了WebGPU与实时流体仿真的相关背景,并简要概括了本文的主要工作。本文后续章节概要如下:
    
    第二章详细介绍了流体模拟的当前研究现状与工作总结,包括粒子法、网格法以及粒子-网格混合法。
    
    第三章将详细介绍基于位置的不可压缩流体的密度约束求解算法,以及边界体积贴图这种隐式边界条件算法。
    
    第四章将介绍用于改善流体细节的三种非压强力,它们分别是表面张力、涡量补偿力与人工粘性力。
    
    第五章将介绍基于空间网格划分的邻域粒子并行查找算法,其中又进一步介绍了数组前缀和并行计算方法及其GPU实现优化细节。
    
    第六章将介绍流体屏幕空间渲染算法,用于将流体粒子实时渲染。
    
    第七章详细介绍了WebGPU架构与重要API功能,然后阐述了流体仿真系统的算法实现方案,最后构建多个实际场景证明系统的可靠性。
    
    第八章总结全文介绍的算法与工作,并分析了本文仿真框架存在的不足与限制,最后展望了浏览器端实时流体仿真的广阔应用前景。