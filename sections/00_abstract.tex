\MakeAbstract{
    实时流体仿真是计算机图形学领域的重要研究课题,但在Web端进行流体仿真面临一系列挑战。传统的方法在追求稳定性的同时,往往牺牲计算效率,难以在性能有限的浏览器上实时运行。然而,WebGPU的出现为Web端流体仿真带来了新的机遇。作为一种新兴的图形渲染API,WebGPU通过全新的架构设计和接近底层硬件的接口,为图形应用提供了更高效的性能优化手段。本文旨在探索基于WebGPU的实时流体仿真方法,突破传统Web端流体仿真的困境,并为各个领域带来更便捷和强大的流体仿真工具。

    本文采用了一系列创新方法来实现基于WebGPU的实时流体仿真。首先,采用基于位置的流体模拟方法,通过迭代求解流体的密度约束,确保了模拟过程中流体的不可压缩性。其次,使用隐式边界条件表示方法来实现固液耦合,将固体边界对流体粒子的影响预计算到三维纹理中,提高了计算效率。第三,添加非压强力模型,包括表面张力、涡量补偿力和人工粘性力,以改善流体细节并提高模拟的真实感。第四,实现了完全并行化的高效邻近粒子搜索算法,通过空间网格划分和计算排序等技术加速搜索过程。最后,采用屏幕空间流体渲染方法,并设计了窄域滤波器用于深度图平滑,在重建流体表面的同时仍能够保留流体轮廓细节。

    通过多个复杂的流体仿真场景,本文验证了所提出方法的稳定性和高效性。实验结果表明,我们的流体仿真系统能够在性能有限的集成显卡上实现实时仿真,并达到30FPS以上的帧率。这证明了本文方法在计算效率方面的优越性和可行性,为Web端实时流体仿真提供了切实可行的解决方案。
}{WebGPU,实时流体仿真,平滑粒子流体动力学,基于位置的流体,屏幕空间渲染}

\MakeAbstractEng{
    Real-time fluid simulation is an important research topic in the field of computer graphics. However, conducting fluid simulation on the web faces a series of challenges. Traditional methods often sacrifice computational efficiency in pursuit of stability, making it difficult to achieve real-time performance on limited-performance browsers. However, WebGPU brings new opportunities for web-based fluid simulation. As an emerging graphics API, WebGPU provides more efficient performance optimization methods for graphics applications through a new architecture design and interfaces close to the underlying hardware. This paper aims to explore real-time fluid simulation methods based on WebGPU, overcome the limitations of traditional web-based fluid simulation, and provide more convenient and powerful fluid simulation tools for various fields.

    This paper adopts a series of innovative methods to achieve real-time fluid simulation based on WebGPU. Firstly, a position-based fluid simulation method is used to iteratively solve the density constraints of the fluid, ensuring the incompressibility of the fluid during the simulation process. Secondly, an implicit boundary representation method is used to realize solid-liquid coupling. The influence of solid boundaries on fluid particles is precomputed into a 3D texture, improving computational efficiency. Thirdly, non-pressure force models, including surface tension, vorticity confinement, and artificial viscosity, are added to improve fluid details and enhance simulation realism. Fourthly, a fully parallelized and efficient neighbor particle search algorithm is implemented, accelerating the search process through techniques such as spatial grid partitioning and counting sort. Finally, a screen-space fluid rendering method is employed, and a narrow-band filter is designed for depth map smoothing, allowing for the reconstruction of fluid surfaces while preserving fluid contour details.

    Through multiple complex fluid simulation scenarios, this paper validates the stability and efficiency of the proposed methods. Experimental results demonstrate that our fluid simulation system can achieve real-time simulation on integrated graphics processing unit of limited-performance, achieving frame rates of over 30 FPS. This demonstrates the superiority and feasibility of the proposed methods in terms of computational efficiency, providing a practical solution for real-time web-based fluid simulation.
}{WebGPU, Real-Time Fluid Simulation, SPH, PBF, Screen Space Rendering}
